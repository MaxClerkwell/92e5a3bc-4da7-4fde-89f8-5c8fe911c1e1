\documentclass{dcbl/challenge}

\setdoctitle{Combinational Logic Circuits}
\setdocauthor{Stephan Bökelmann}
\setdocemail{sboekelmann@ep1.rub.de}
\setdocinstitute{AG Physik der Hadronen und Kerne}


\begin{document}

Boolean logic helps us abstracting logic circuits.
Using the combination of several of these elements, we can build more sophisticated circuits and mappings.
A mapping is a function that maps one set of values to another set of values.
An addition for example is a mapping between two sets of values onto a third set of values, where the third set consists of all possible sums of two values from the first two sets.
Using combinational logic circuits, we can build these mappings.


\section*{Exercises}
\begin{aufgabe}
    In the November of 1937, Georg Stibitz created the first digital adder, using relais, switches and lamps. Take a look at this simulation of a 1-bit adder: \url{https://circuitverse.org/simulator/embed/1-bit-adder-62af8084-2923-41a2-bb79-b3151d12d088}. 
    Write down the boolean equation for this circuit. 
\end{aufgabe}

\begin{aufgabe}
    Using this idea of a 1-bit adder, write down the boolean equation for the following circuits:
    \begin{enumerate}
        \item 2-bit adder
        \item 4-bit adder
        \item 1-bit subtractor
        \item 2-bit multiplier
    \end{enumerate}
\end{aufgabe}

\begin{aufgabe}
    Implement a 2-bit adder using only combinational logic and the Circuit-Verse simulator. \url{https://circuitverse.org/}.
\end{aufgabe}

\begin{aufgabe}
    Write down an elaborate argument, about which mappings can, and which mappings can not be build with this kind of combinational logic.
\end{aufgabe}

\section*{Anmerkungen}
\begin{enumerate}
    \item Georg Stibitz on the computer history museum: \url{https://www.computerhistory.org/revolution/birth-of-the-computer/4/85}
    \item Electronic Tutorials on combinational logic: \url{https://www.electronics-tutorials.ws/combination/comb_1.html}
\end{enumerate}

\end{document}
